Samozřejmě jen ty nejvtipnější z nejvtipnějších matematických humoresek se zde
nacházejí. Ty ostatní (méně vtipné leč stále výtečné) najdete v ostatních
kapitolách.



\section{Láska moment}
\label{sec:laska-moment}
\[
 \forall \epsilon > 0
\]
Pro všechna srdíčka 0. Když si člověk definuje $\epsilon >$ jako \uv{srdíčk} tak
to celé potom dá \uv{srdíčko}. awwww


\section{OMG koitus?!?!}
\label{sec:omg-koitus}
\[
 s \in X
\]
Představte si, že je prvek $s$ v množine $X$. Tomu říkám ten pořádný humor. A co
tepr, kdyby byl v reálných číslech?. hmmmm
\[
 s \in \mathbb{R}
\]
Teď si dokonce představte, že definujeme faktoriál na obecné množině a $s$ by
stále bylo prvek této množiny. To by ale znamenalo...
\[
 s \in \mathbb{R}! \text{~nebo rovnou~} s \in X!
\]
Nezbývá než činit jak nám káže matematika.
